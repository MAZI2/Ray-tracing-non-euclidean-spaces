\documentclass[titlepage]{article}
\usepackage{algpseudocode}
\usepackage{algorithm}
\usepackage{graphicx} % Required for inserting images
\usepackage[utf8]{inputenc}
\usepackage{setspace}
\usepackage{amssymb}
\usepackage{amsmath}
\usepackage{listings}
\usepackage{matlab-prettifier}
\usepackage{url}
\usepackage{hyperref}
\usepackage{float}
\usepackage{caption}
\usepackage{subcaption}
\graphicspath{ {./Images/} }

\title{Sledenje žarku v neevklidskih prostorih}
\author{Blaž Bergant, Martin Jereb, Matjaž Pogačnik}
\date{Matematično modeliranje, maj 2024}

\begin{document}

\maketitle
\newpage
\tableofcontents
\newpage

\section{Uvod}
Algoritem sledenja žarku (\textit{ang. Ray Tracing}) je priljubljen algoritem, uporabljen v 
računalniški grafiki, ki simulira realistično osvetlitev prizorov. Temelji na ideji sledenja 
žarka svetlobe po prizoru, izračunavanju njegovega preseka z objekti in odbijanju v drugo smer. 
Običajno ta algoritem uporabljamo v običajnem evklidskem prostoru, saj predstavlja resnični svet. 
Vendar pa lahko algoritem implementiramo tudi v drugih (neevklidskih) prostorih, da dobimo 
zanimive vizualne rezultate. To bo glavna naloga tega projekta.

\section{Cilji}
Cilj za projektno nalogo, nam je bil narediti program, v katerem uporabnik lahko eksperimentira 
kako zgledajo (kako potujejo žarki) v prostorih ki niso evklidski, in primerjavo teh prostorov z 
evklidskim prostorom. Želeli smo si, da bi bilo dodajanje novih prostorov in novih objektov kar 
se le da enostavno in da na slike nebi bilo potrebno dolgo čakati.
\section{Sledenje žarkom}
Predstavljajte si, da ste na vaši najljubši plaži, ki ima razgled na ocean. Ko se ozrete na barko,
ki potuje proti sončnemu zahodu, boste poleg tega da se barka oddaljuje opazili tudi, da se "pogreza" 
v tla.

To je ravno zaradi potovanja žarkov. Ker smo v evklidskem prostoru, bo žarek potoval naravnost in nas 
zaradi ukrivljenosti Zemlje tako, po določeni razdalji, ne bo več dosegel. Sedaj, pa si predstavljajmo, 
da bi lahko barko kljub oddaljevanju še vedno videli celotno. Ta posebna zmožnost, bi nam bila 
omogočena v sferičnem neevklidskem prostoru.

\begin{figure}[H]
    \centering
    \includegraphics[width=0.5\linewidth]{Images/potovanje_zarkov.png}
    \caption{Primerjava evklidskega in sferičnega prostora [Vir: G. Kovač]}
    \label{Slika:Primerjava evklidskega in sferičnega prostora}
\end{figure}

Kaj pa pravzaprav je neevklidski prostor?
Neevklidski prostor je preprosto prostor, ki ne sledi pravilom evklidske geometrije. Torej prostor ni raven ampak je ukrivljen, vsota kotov v trikotniku je večja ali pa manjša od 180 stopinj... Neevklidski prostor delimo na več različnih geometrij med katerimi so recimo hiperbolična geometrija, eliptična geometrija, sferična geometrija, projektivna geometrija...

\section{Implementacija osnovnega algoritma ray tracing}
\subsection{Objekti v sceni}
Posamični objekti v prostoru so podani z enačbo v obliki:
\[ o(x,y,z): \mathbb{R}^3 \to R =0 \]
Za kroglo je to enačba: 
\[ (\vec{P} - \vec{C})^2 - r^2 = 0 \]
kjer je \(\vec{P}\) točka na krogli, \(\vec{C}\) središče krogle in \(r\) polmer krogle. \newline
Za ravnino pa enačba 
\[ \vec{n} \cdot \vec{P} - \vec{n} \cdot \vec{C} = 0 \]
kjer je \(\vec{n}\) normala ravnine, \(\vec{P}\) pa koeficienti normale ravnine, \(d\) pa konstanta.

\subsection{Prostori}
Vsak prostor implementira funkcijo, v obliki
\[ s(t): \mathbb{R} \to \mathbb{R}^3 = [x, y, z]^T \]
ki zna žarek podan z točko \(\vec{T}\) in smerjo \(\vec{v}\) premakniti za \(t\) korakov v smeri \(\vec{v}\).
V primeru evklidskega prostora je to preprosto premikanje za \(t\) korakov v smeri \(\vec{v}\), 
\[ \vec{T}_{n+1} = \vec{T}_{n} + t \cdot \vec{v} \]
Prostora flat torus in 2-sphere sta podrobneje opisana kasneje.

\subsection{Algoritem}
Želeli smo si splošen način, kako lahko premikamo žarek skozi katerikoli prostor. Tako smo zasnovali algoritem, ki temelji na iskanju ničle funkcije vsakega posameznega objekta \( o = (x, y, z) \), ko v njo vstavimo pozicijo žarka ob podanem "času" \( t \). Algoritem torej išče parameter \( t \) za katerega velja:
\[ f = o(s(t)) = 0 \]

\bigskip
\begin{enumerate}

\item Začnemo v točki \( T_{0} \) - pozicija kamere, iz koder pošiljamo žarke v smeri, določene z Field of view (FOV) kamere, ter resolucijo slike. 
\item Žarek v vsakem koraku premaknemo za \( h \) v smeri \(\vec{v}\).
\item V poljubnem prostoru premikamo žarek za nek korak, kjer vsakič preverimo vrednost funkcije \( f \), ki jo vrne vsak objekt.
\item Če se v neki točki funkciji predznak spremeni, vemo da smo v tem koraku vstopili v objekt. Razpolovimo korak in ga ponovimo.
\item Če je v neki točki aproksimacija odvoda \( f \) pozitivna, vemo da se od objekta oddaljujemo. Mogoče smo ga z korakom preskočili, ali pa smo odšli mimo. Razpolovimo korak in ga ponovimo.
\item Postopek razpolavljanja ponavljamo, dokler korak ni manjši kot \( \varepsilon \).

\begin{figure}[H]
    \centering
    \includegraphics[width=0.5\linewidth]{intersect.png}
    \caption{Iskanje natančnejšega presečišča}
    \label{Slika:Iskanje natančnejšega presečišča}
\end{figure}

\item Žarek od točke presečišča pošljemo v smer luči. Če žarek pride do luči, brez da bi se prej sekal z objektom, je točka osvetljena.
\item Glede na barvo objekta in ugotovljeno osvetlenost pobarvamo točko na sliki, ki pripada žarku.
\end{enumerate}

\bigskip
V nadaljevanju smo implementirali algoritem za sledenje žarku na tridimenzionalnem ploščatem torusu \( \mathbb{T}_{pl}^{3} \) in dvodimenzionalni sferi \( \mathbb{S}^2 \) v \( \mathbb{R}^3 \). Algoritem za sledenje žarku v evklidskem prostoru je enak algoritmu za torus, brez preslikave žarka na željeno domeno. Sledenje v smeri vira svetlobe uporablja algoritem za sledenje žarku v evklidskem prostoru, pri tem pa ga pri korakih dolžine \( h \) omejimo na maksimalno število korakov
\[ k_{max}= \left \lceil \frac{\left \|\vec{r}_{l}-\vec{r}_{t} \right \|}{h} \right \rceil \]
kjer je \( \vec{r}_{l} \) krajevni vektor vira svetlobe, \( \vec{r}_{t} \) pa krajevni vektor točke presečišča.

\section{\texorpdfstring{Sledenje žarkom na ploščatem torusu \( \mathbb{T}_{pl}^{3} \)}{Sledenje žarkom na ploščatem torusu}}

\subsection{Kaj je tri dimenzionalni ploščati torus?}
Tri dimenzionalni ploščati torus \( \mathbb{T}^3_{pl} \), je tridimenzionalna mnogoterost v evklidski geometriji. \( \mathbb{T}^3_{pl} \) dobimo tako, da "zlepimo" nasprotne ploskve enotske kocke \([0,1] \times [0,1] \times [0,1] \subset \mathbb{R}^3 \), kjer je \(\mathbb{R}^3\) tridimenzionalni evklidski prostor. Formalno pravimo, da je \( \mathbb{T}^3_{pl} \) kvocient \(\mathbb{R}^3\) z grupo translacij:
\[
(x, y, z) \to (x \pm 1, y, z), \quad (x, y, z) \to (x, y \pm 1, z), \quad (x, y, z) \to (x, y, z \pm 1).
\]

Običajni torusi so ukrivljeni in se jim ukrivljenost po površini spreminja. Ploščati torus pa ima konstantno ukrivljenost v vsaki točki. Poimenovanje ploščat izhaja iz tega, ker je povsod podoben navadni evklidski ravnini. Podobno velja za tridimenzionalni ploščati torus, kjer pri sledenju žarka lahko uporabimo kar osnovno definicijo žarka. Ključna razlika med algoritmom v \( \mathbb{R}^3 \) in \( \mathbb{T}^3_{pl} \) pa je, da bomo tokrat dodali še en korak, ker moramo žarek omejiti na kocko. Dokler se nahajamo znotraj njenih meja, ne naredimo nobenih popravkov. Ko pa ugotovimo presečišče z eno od ploskev, moramo žarek preslikati na nasprotno ploskev. To lahko storimo tako, da preprosto žarek omejimo znotraj domene, izračunamo presežek in ga prištejemo začetku domene.

\begin{figure}[H]
    \centering
    \includegraphics[width=0.5\linewidth]{Images/flat_torus_zrcaljenje.png}
    \caption{Žarek v dvodimenzionalnem ploščatem torusu}
    \label{Slika:Žarek v dvodimenzionalnem ploščatem torusu}
\end{figure}

\subsection{Algoritmi za flat torus}
Ključni vpogled, je da žarek omejimo na domeno. V spodnji funkciji mapToCube je Tn začetni vektor, a je začetek domene, b konec domene.
\begin{algorithm}[H]
\caption{Map to Cube}\label{alg:mapToCube}
\begin{algorithmic}[1]
\Function{mapToCube}{$T_{n}$, $a$, $b$} 
    \State $range \gets b - a$
    \State $q \gets (T_{n} - a) / range$
    \State $fraction \gets q - \text{floor}(q)$
    \State $T_{n+1} \gets a + fraction \cdot range$
    \State \Return $T_{n+1}$
\EndFunction
\end{algorithmic}
\end{algorithm}

Funkciji \textbf{signs} in \textbf{checkIntersect} implementirata funkcije \( f \), \textbf{distance} pa implementira formuli (6) in 
(7). Funkcija \textbf{remap} implementira funkcijo mapToCube.
\begin{algorithm}[H]
    \caption{Sledenje žarku na ploščatem torusu}
\begin{algorithmic}
    \Function{traceRayFlatTorus}{$T_{0}$, $\vec{v}$, step, maxIt, objects, a, b, $\varepsilon$}

    \State $T_{n}$, $T_{n+1}$ $\gets$ $T_{0}$
    \State signs $\gets$ \textbf{signs}($T_{n}$, objects)
    \Comment{izračunamo začetne predznake}
    \\
  \While{true}
    \If{step == maxIt}
      \State \Return{-1}
    \ElsIf{\textbf{distance}($T_{n}$, objects) $<$ $\varepsilon$}
      \State \Return{$T_{n}$}
    \EndIf
    \State $T_{n+1}$ $\gets$ $T_{n}+step \cdot \vec{v}$
    \State $T_{n+1}$ $\gets$ \textbf{remap}($T_{n+1}$, a, b)
    \\
    \Comment{naredimo korak, preslikamo na željeno domeno}
    \\
    \State signs $\gets$ \textbf{checkIntersects}($T_{n+1}$, objects, signs)
    \\
    \Comment{preverimo, če smo sekali objekt}
    \\
    \Comment{če da, razpolovimo korak}

    \If{object != NULL}
        \State step $\gets$ step/2
    \Else
      \State $T_{n}$ $\gets$ $T_{n+1}$
    \EndIf
  \EndWhile
\EndFunction
\end{algorithmic}
\end{algorithm}

Ko program poženemo, dobimo lep vizualni rezultat.
\begin{figure}[H]
    \centering
    \includegraphics[width=0.5\linewidth]{Images/flat_torus.png}
    \caption{Rezultat "ray tracinga" za tri dimenzionalni ploščati torus (10 preslikanj)}
    \label{Slika:Rezultat "ray tracinga" za tri dimenzionalni ploščati torus 1}
\end{figure}
Na sliki lahko opazimo, da so elementi simetrično preslikani, prav tako pa lahko vidimo, da so tudi različno osenčeni. Opazmo lahko tudi, veliko "črnine" (void). Črna barva pomeni, da se tam žarki ne sekajo z nobenim objektom. Do tega pride, ker smo na zgornji sliki pustili manj preslikanj (okrog 10), da je bila slika hitreje generirana. Če pa pustimo več preslikanj, recimo 500, pa dobimo sliko, kjer črnine skoraj ni.

\begin{figure}[H]
    \centering
    \includegraphics[width=0.5\linewidth]{Images/flat_torus_more.png}
    \caption{Rezultat "ray tracinga" za tri dimenzionalni ploščati torus (500 preslikanj)}
    \label{Slika:Rezultat "ray tracinga" za tri dimenzionalni ploščati torus 2}
\end{figure}

\section{\texorpdfstring{Sledenje žarkom na dvodimenzionalni sferi \( \mathbb{S}^2 \)}{Sledenje žarkom na dvodimenzionalni sferi}}
Žarek na dvodimenzionalni sferi bo potoval po najravnejši poti, tj. geodetki. Geodetka je krivulja po sferi, kjer zahtevamo, da je
pospešek v smeri normale tangentne ravnine 0. Majhen korak po geodetki je v \( uv \) ravnini opisan s sistemom dveh diferencialnih enačb drugega reda.
\begin{equation}
    \begin{split}
        &\frac{d^{2}u}{dt^{2}}-\cos(u)\sin(u)\frac{dv}{dt}\frac{dv}{dt}=0 \\
        &\frac{d^{2}v}{dt^{2}}+2\cot(u)\frac{du}{dt}\frac{dv}{dt}=0
    \end{split}
\end{equation}

Po krivulji se bomo v nadaljnjih algorimih premikali s pomočjo aproksimacijskih metod, zato moramo sistem DE drugega reda preoblikovati v
sistem DE prvega reda. Za to uvedemo 4 nove spremenljivke
\begin{equation}
\begin{split}
    &y_{1}=u, \quad y_{2}=\frac{dy_{1}}{dt}, \\
    &y_{3}=v, \quad y_{4}=\frac{dy_{3}}{dt}
\end{split}
\end{equation}
Dobimo sistem 4 DE prvega reda
\begin{equation} \label{e:geoSys}
\begin{split}
    &\frac{dy_{1}}{dt}=y_{2} \\
    &\frac{dy_{2}}{dt}=\cos(y_{1})\sin(y_{1})y^{2}_{4} \\
    &\frac{dy_{3}}{dt}=y_{4} \\
    &\frac{dy_{4}}{dt}=-2\cot(y_{1})y_{2}y_{4}
\end{split}
\end{equation}
Ker se bomo premikali po \( uv \) ravnini, moramo koordinate v kartezičnem koordinatnem sistemu transformirati v parametra \( u \) in \( v \). Sfera v \(\mathbb{R}^3\) je
parametrizirana kot
\begin{equation} \label{e:toXYZ}
    \begin{split}
        &X=R\cos(v)\sin(u) \\
        &Y=R\sin(v)\sin(u) \\
        &Z=R\cos(u)
    \end{split}
\end{equation}
\( u \) lahko dobimo kot
\begin{equation}
        u=\cos^{-1} \left( \frac{X}{R} \right)
\end{equation}
za \( v \) pa uporabimo prvi dve enačbi
\begin{equation} \label{e:toU}
    \begin{split}
        &\frac{R\sin(u)Y}{R\sin(u)X}=\frac{\sin(v)}{\cos(v)} \\
        &v=\tan^{-1} \left(\frac{Y}{X} \right)
    \end{split}
\end{equation}

Za sledenje žarku po geodetki, si moramo poleg začetne točke izbrati še začetno smer
\(\left( \frac{du}{dt}, \frac{dv}{dt} \right) \). Tako kot pri ostalih prostorih v nalogi, bomo uporabili "FOV" način
potovanja žarkov iz kamere. Od tu dalje imamo svobodo pri izbiri sfer, po katerih bodo žarki potovali. V tej nalogi smo za
vsak žarek izračunali središče sfere z najnižjo \( z \) koordinato in polmerom \( R \), žarki pa bodo vedno "zavijali" navzdol, torej se bodo premikali po poldnevnikih.

\begin{figure}[H]
\centering
\begin{minipage}{.5\textwidth}
  \centering
  \includegraphics[width=0.8\linewidth]{Images/rays_top.png}
  \captionof{figure}{Potovanje žarkov od zgoraj}
  \label{fig:test1}
\end{minipage}%
\begin{minipage}{.5\textwidth}
  \centering
  \includegraphics[height=0.77\linewidth]{Images/rays_side.png}
  \captionof{figure}{Potovanje žarkov s strani}
  \label{fig:test2}
\end{minipage}
\end{figure}

Središča sfer žarkov se bodo torej nahajala na sferi s polmerom \( R \) in središem v kameri \( T_{0} \). Za različne smeri žarkov \(\vec{v}_{i} \) je bilo središče \( C_{i} \) izračunano po sledečem postopku

\bigskip

Izračunamo vektorski produkt med navpičnim vektorjem in smerjo \( \vec{v}_{i} \). Dobljeni vektor je normala na ravnino krožnice z iskanim središčem, ki gre skozi \( T_{0} \)
\begin{equation} \label{e:sphC1}
    \vec{n}_{i}=(0, 0, -1) \times \vec{v}_{i}
\end{equation}
Vektor, ki od \( T_{0} \) kaže proti iskanemu središču bo ležal v tej ravnini, poleg tega pa bo pravokoten na \( \vec{v}_{i} \). Ponovno uporabimo vektorski
produkt
\begin{equation}\label{e:sphC2}
    \vec{c}_{i}= \vec{v}_{i} \times \vec{n}_{i}
\end{equation}
Krajevni vektor središča nato dobimo tako, da se premaknemo za polmer \( R \) v smeri \( \vec{c}_{i} \)
\begin{equation}\label{e:sphC3}
    \vec{r}_{ci}=\frac{\vec{c}_{i}}{\left \| \vec{c}_{i}\right \|} \cdot R + T_{0}
\end{equation}

Za korak po geodetki bomo \( T_{0} \) najprej premaknili za vektor \( \vec{r}_{ci} \), tako da bo središče sfere v koordinatnem izhodišču. Nato naredimo
korak, novo točko \( T_{i1} \) pa premaknemo nazaj za vektor \( \vec{r}_{ci} \).
Ker želimo potovati po poldnevnikih, bomo za začetno smer \(\left( \frac{du}{dt}, \frac{dv}{dt} \right) \) izbrali \(\left( \pm1, 0 \right) \). Korak po poldnevniku bo v \( xy \) ravnini tako že pravilno obrnjen, določiti pa moramo predznak premika po \( u \) glede na to, ali se premikamo v pravo smer \( \vec{v}_{i} \). To preverimo s skalarnim produktom
\bigskip
\newline
Naredimo korak v smeri \(\left( 1, 0 \right) \) in označimo dobljeno točko s \( T_{t} \). Vektor v smeri od \( T_{0} \) do \( T_{t} \) označimo z \( \vec{v}_{t} \).
Izračunamo skalarni produkt
\begin{equation} \label{e:dirCorr}
    a= \vec{v}_{i} \vec{v}_{t}^T
\end{equation}
Če se premikamo v pravi smeri, bo \( a > 0 \), drugače moramo za začetno smer izbrati negativni predznak.
\bigskip
\newline
V nadaljevanju se premikamo po geodetki z eno izmed aproksimacijskih metod. V splošnem lahko za take sisteme uporabimo Eulerjevo metodo
\begin{equation} \label{e:euler}
    \begin{split}
        &t_{n+1}=t_{n}+h \\
        &\vec{y}_{n+1}=\vec{y}_{n}+h \cdot \vec{f}(t_{n}, \vec{y}_{n})
    \end{split}
\end{equation},
problem pa se pojavi pri \( \cot(y_{1}) \) v eni izmed enačb, ki ima pole pri
\( k\pi \), \( k \in \mathbb{Z} \). Da se temu izognemo ne smemo uporabljati fiksne dolžine koraka, temveč moramo uporabiti adaptivne metode, kot je DOPRI5,
ki z uporabo večih metod estimira napako približka, in temu primerno prilagodi korak.
\bigskip
\newline
Algoritmi, ki implementirajo opisane metode so predstavljeni v naslednjem poglavju.
\newpage

\section{Algoritmi za 2-sphere}
Funkcija \textbf{sphereCenter} implementira formule \eqref{e:sphC1}, \eqref{e:sphC2}, \eqref{e:sphC3}, \textbf{initializeSphere} formule \eqref{e:geoSys}, \eqref{e:toU}, \eqref{e:toXYZ}, \textbf{uvToVec} pa formulo (20). Funkcija \textbf{DOPRI5} predstavlja implementacijo DOPRI5, ki naredi po en korak glede na trenutno velikost koraka, in skupaj s približkom vrne tudi novo velikost koraka. Tu lahko uporabimo tudi katero drugo adaptivno funkcijo. Ostale funkcije so opisane v
razdelku 
\begin{algorithm}
    \caption{Sledenje žarku na sferi \(\mathbb{S}^{2}\)}
\begin{algorithmic}
    \Function{traceRayNonEuclidean}{$T_{0}$, $\vec{v}$, step, maxIt, objects, R, $\varepsilon$}

    \State $T_{n}$, $T_{n+1}$ $\gets$ $T_{0}$
    \State signs $\gets$ \textbf{signs}($T_{n}$, objects)
    \Comment{izračunamo začetne predznake}
    \State center, I
    \\
  \While{true}
    \If{step == maxIt}
    \State \Return{-1}
    \ElsIf{step == 0}
      \State center $\gets$ \textbf{sphereCenter}($T_{0}$, d, R)
      \\
      \Comment{žarku poiščemo središče sfere}
      \State $T_{m}$ $\gets$ $T_{0} - \hbox{center}$
      \Comment{sfero premaknemo v $(0, 0)$}
      \State $\vec{y}_{n}$ $\gets$ \textbf{initializeSphere}($T_{0}$, R)
      \Comment{pripravimo začetni $\vec{y}$}
    \EndIf
    \\
    \State $\vec{y}_{n+1}, \hbox{step}$ $\gets$ \textbf{DOPRI5}($\vec{y}_{n}$, step)
    \Comment{korak po geodetki}
    \State $T_{n+1}$ $\gets$ \textbf{uvToVec}($\vec{y}_{n+1}$, R) + center
    \\
    \State object $\gets$ \textbf{checkIntersects}($T_{n+1}$, objects, signs)
    \\
    \Comment{preverimo, če smo sekali objekt}
    \\
    \Comment{če da, poiščemo natančno presečišče}

    \If{object != NULL}
      \State I $\gets$ \textbf{findIntersection}($T_{n}$, object, signs, $\vec{y}_{n}$, step, R, center, $\varepsilon$)
      \State \Return{I}
    \Else
      \State $\vec{y}_{n}$ $\gets$ $\vec{y}_{n+1}$
      \State $T_{n}$ $\gets$ $T_{n+1}$
    \EndIf
  \EndWhile
\EndFunction
\end{algorithmic}
\end{algorithm}

\begin{algorithm}[H]
    \caption{Iskanje natančnejšega presečišča}
\begin{algorithmic}
    \Function{findIntersection}{$T_{0}$, object, signs, $\vec{y}_{n}$, step, R, center, $\varepsilon$}

    \State $T_{n}$, $T_{n+1}$ $\gets$ $T_{0}$
    \\
  \While{true}
    \If{\textbf{distance}($T_{n}$, object) $<$ $\varepsilon$}
      \State \Return{$T_{n}$}
    \EndIf
    \State $\vec{y}_{n+1}, \hbox{step}$ $\gets$ \textbf{euler}($\vec{y}_{n}$, step)
    \State $T_{n+1}$ $\gets$ \textbf{uvToVec}($\vec{y}_{n+1}$, R) + center
    \\
    \Comment{korak po geodetki z Eulerjevo metodo}
    \State object $\gets$ \textbf{checkIntersects}($T_{n+1}$, object, signs)
    \\
    \Comment{preverimo, če smo sekali objekt}
    \\
    \Comment{če da, razpolovimo korak}

    \If{object != NULL}
        \State step $\gets$ step/2
    \Else
      \State $\vec{y}_{n}$ $\gets$ $\vec{y}_{n+1}$
      \State $T_{n}$ $\gets$ $T_{n+1}$
    \EndIf
  \EndWhile
\EndFunction
\end{algorithmic}
\end{algorithm}

\section {Primerjava metode sledenje žarku v evklidskem in neevklidskem prostoru}


\begin{itemize}
\item sledi pravilom evklidske geometrije
\item prostor je raven (žarki potujejo naravnost)
\item koti v trikotniku so skupaj 180 stopinj
\item nam najbolj poznan
\end{itemize}

\begin{itemize}
\item ne sledi pravilom evklidske geometrije
\item prostor je ukrivljen (žarki ni nujno, da potujejo naravnost)
\item koti v trikotniku skupaj več / manj od  180 stopinj
\end{itemize}

\section{Razdelitev dela}
\begin{itemize}
  \item Blaž Bergant: Zasnova programa, implementacija osnovnega raytracing algoritma, izpeljava določenih formul za hitrejše delovanje.
  \item Martin Jereb: Priprava poročila, razlaga in implemenacija prostora flat torus, priprava predstavitve.
\item Matjaž Pogačnik: Implementacija in izpeljava algoritmov za 2Sphere, priprava poročila in generiranje slik.
\end{itemize}

\section{Viri}
\begin{itemize}
  \item Kovač, G. (2023). \textit{Sledenje žarku v neevklidskih prostorih} (Diplomska naloga). Ljubljana: [G. Kovač].
  \item Kovač, G. (2024). \textit{Ray tracing in Non-euclidean spaces}. Ljubljana: [G. Kovač].
  \item Zalar, A. \textit{Mathematical Modelling, Lecture Notes} (2024). Ljubljana: [A. Zalar].
  \item \textit{Non-Euclidean geometry}. (2024). Pridobljeno 24.5.2024 s spletne strani: \url{https://en.wikipedia.org/wiki/Non-Euclidean_geometry.}
\end{itemize}

\section{Priloga}
\begin{itemize}
  \item Koda je na voljo na spletnem repozitoriju, na \href{https://github.com/MAZI2/Ray-tracing-non-euclidean-spaces}{povezavi}.
  \item Spletna predstavitev pa je na voljo \href{https://docs.google.com/presentation/d/1NP8gkPzV8rE2ToBoUAP4b7w2yMkAWgCmlsRPMn_5Ahc/edit?usp=sharing}{tukaj}.
\end{itemize}
\end{document}

