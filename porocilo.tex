\documentclass[titlepage]{article}
\usepackage{graphicx} % Required for inserting images
\usepackage[utf8]{inputenc}
\usepackage{setspace}
\usepackage{amssymb}
\usepackage{amsmath}
\usepackage{listings}[style=Matlab-editor]
\usepackage{matlab-prettifier}
\usepackage{url}
\usepackage{float}

\title{Sledenje žarku v neevklidskih prostorih}
\author{Blaž Bergant, Martin Jereb, Matjaž Pogačnik}
\date{Matematično modeliranje, maj 2024}
\begin{document}

\doublespacing

\maketitle
\newpage
\tableofcontents
\newpage


\section{Uvod}
Algoritem sledenje žarku (\textit{ang. Ray Tracing}) je priljubljen algoritem, uporabljen v računalniški grafiki, ki simulira realistično osvetlitev prizorov. Temelji na ideji sledenja žarka svetlobe po prizoru, izračunavanju njegovega preseka z objekti in odbijanju v drugo smer. Običajno ta algoritem uporabljamo v običajnem evklidskem prostoru, saj predstavlja resnični svet. Vendar pa lahko algoritem implementiramo tudi v drugih (neevklidskih) prostorih, da dobimo zanimive vizualne rezultate. To bo glavna naloga tega projekta.

\section{Cilji}
Cilji za to projektno nalogo so implementacija "osnovnega" algortima sledenje žarkom v evklidskem prostoru, njegova nadgradnja, da deluje tudi v neevklidskem prostoru na ploščatem torusu in 2-sferi ter primerjava delovanja v evklidskem in neevklidskem prostoru.

\section{Sledenje žarkom}

Predstavljajmo si, da smo na plaži v naši najljubši obalni turistični destinaciji in gledamo, kako bližnja barka pluje v sončni zahod. Nam, na obali zgleda, da se poleg oddaljevanja tudi "pogreza" v tla. Zakaj pa je temu tako? \\

To je ravno zaradi potovanja žarkov. Ker smo v evklidskem prostoru, bo žarek potoval naravnost in nas zaradi ukrivljenosti Zemlje tako, po določeni razdalji ne bo več dosegel. Sedaj, pa si predstavljajmo, da bi lahko barko kljub oddaljevanju še vedno videli celotno. Ta posebna zmožnost, bi nam bila omogočena v neevklidskem prostoru. In sicer še bolj natančno, v sferičnem neevklidskem prostoru.\\

\begin{figure}
    \centering
    \includegraphics[width=0.5\linewidth]{Screenshot 2024-05-25 at 00.37.45.png}
    \caption{Primerjava [Vir: G. Kovač]}
    \label{Slika:enter-label}
\end{figure}

Kaj pa pravzaprav je neevklidski prostor?
Neevklidski prostor je preprosto prostor, ki ne sledi pravilom evklidske geometrije. Torej prostor ni raven ampak je ukrivljen, vsota kotov v trikotniku je večja ali pa manjša od 180 stopinj... Neevklidski prostor delimo na več različnih geometrij med katerimi so recimo hiperbolična geometrija, eliptična geometrija, sferična geometrija, projektivna geometrija... 



\section {Implementacija osnovnega algoritma ray tracing}
Implementirali smo osnovni algoritem sledenja žarkov. Naš program je sposoben upodobiti preprost prizor z osnovnim senčenjem. 
A SMO TO KEJ DODAL AL NE ???Vse druge razširitve, kot so odsevi in mehko senčenje, so izbirne, saj niso v ospredju tega projekta. ????? TUKI JE TREBA OPISAT POSTOPEK KAKO ZA PRVO VPRAŠANJE 

\subsection{Metoda Intersection}
Razlaga kode 


\section{Implementacija algoritma sledenje žarkom za tri dimenzionalni ploščati torus} 

\subsection{Kaj je tri dimenzionalni ploščati torus?}
Tri dimenzionalni ploščati torus \( \mathbb{T}^3_{pl} \), je tridimenzionalna mnogoterost v evklidski geometriji. \( \mathbb{T}^3_{pl} \) dobimo tako, da "zlepimo" nasprotne ploskve enotske kocke \([0,1] \times [0,1] \times [0,1] \subset \mathbb{R}^3 \), kjer je \(\mathbb{R}^3\) tridimenzionalni evklidski prostor. Formalno pravimo, da je \( \mathbb{T}^3_{pl} \) kvocient \(\mathbb{R}^3\) z grupo translacij:
\begin{equation}
(x, y, z) \to (x \pm 1, y, z), \quad (x, y, z) \to (x, y \pm 1, z), \quad (x, y, z) \to (x, y, z \pm 1).
\end{equation}

Običajni torusi so ukrivljeni in se jim ukrivljenost po površini spreminja. Ploščati torus pa ima konstantno ukrivljenost v vsaki točki. Poimenovanje ploščat izhaja iz tega, ker je povsod podoben navadni evklidski ravnini. Podobno velja za tridimenzionalni ploščati torus, kjer pri sledenju žarka lahko uporabimo kar osnovno definicijo žarka. Ključna razlika med algoritmom v \( \mathbb{R}^3 \) in \( \mathbb{T}^3_{pl} \) pa je, da bomo tokrat dodali še en korak, ker moramo žarek omejiti na enotsko kocko. Dokler se nahajamo znotraj njenih meja, ne naredimo nobenih popravkov. Ko pa ugotovimo presečišče z eno od ploskev, moramo žarek preslikati na nasprotno ploskev. To lahko storimo tako, da preprosto od žarka odštejemo normalo ploskve, na kateri smo.

\begin{figure}[H]
    \centering
    \includegraphics[width=0.5\linewidth]{Screenshot 2024-05-26 at 02.02.11.png}
    \caption{Žarek v dvodimenzionalnem ploščatem torusu}
    \label{fig:enter-label}
\end{figure}



\subsection{Razlaga kode}

Ključni vpogled, je da žarek omejimo, kar lahko v programskih jezikih storimo preprosto z operatorjem modulo. Rešitev torej dobimo, ko žarek preslikamo z modulo 1 in tako rezultat omejimo na enotsko kocko.
        
\begin{align*}
x &= \left( \text{x} + \text{korak} \cdot \text{smer x }\right) \mod 1 \\
y &= \left( \text{y} + \text{korak} \cdot \text{smer y }\right) \mod 1 \\
z &= \left( \text{z} + \text{korak} \cdot \text{smer z }\right) \mod 1
\end{align*}

\section{Implementacija algoritma sledenje žarkom za 2-sphere}
Suggestion: first just try to plot one geodesic
using the method described above and then move to ray tracing.

\subsection{Kaj je 2-sphere?}

\subsection{Razlaga kode}
PREDZADNJE VPRAŠANJE 

\section {Primerjava metode sledenje žarku v evklidskem in neevklidskem prostoru}

SLIKA ZA PRIMERJAVO / ANIMACIJE WHATEVER


sledi pravilom evklidske geometrije
prostor je raven (žarki potujejo naravnost)
koti v trikotniku so skupaj 180 stopinj
nam najbolj poznan

ne sledi pravilom evklidske geometrije
prostor je ukrivljen (žarki ni nujno, da potujejo naravnost)
koti v trikotniku skupaj več / manj od  180 stopinj


\section{Razdelitev dela}
\begin{itemize}
  \item Blaž Bergant: DOPIŠI SAM
  \item Martin Jereb: ogrodje poročila, implementacija razreda Torus, priprava PowerPoint predstavitve
  \item Matjaž Pogačnik: DOPIŠI SAM 
\end{itemize}


\section{Viri}
\begin{itemize}
  \item Kovač, G. (2023). \textit{Sledenje žarku v neevklidskih
prostorih} (Diplomska naloga). Ljubljana: [G. Kovač].
  \item Kovač, G. (2024). \textit{Ray tracing in Non-euclidean spaces}. Ljubljana: [G. Kovač].
  \item Zalar, A. \textit{Mathematical Modelling, Lecture Notes} (2024). Ljubljana: [A. Zalar].
    \item \textit{Non-Euclidean geometry}. (2024). Pridobljeno 24.5.2024 s spletne strani \url{https://en.wikipedia.org/wiki/Non-Euclidean_geometry.}
\end{itemize}

\section{Priloga}
TUKI PRIDE KODA AL PA SLIKE AL KARKOL K NE BO PASAL V POROČILO

\end{document}

