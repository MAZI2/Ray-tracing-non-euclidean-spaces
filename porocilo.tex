\documentclass[titlepage]{article}
\usepackage{graphicx} % Required for inserting images
\usepackage[utf8]{inputenc}
\usepackage{setspace}
\usepackage{amssymb}
\usepackage{amsmath}
\usepackage{listings}[style=Matlab-editor]
\usepackage{matlab-prettifier}


\title{Sledenje žarku v neevklidskih prostorih}
\author{Blaž Bergant, Martin Jereb, Matjaž Pogačnik}
\date{Matematično modeliranje, maj 2024}
\begin{document}

\doublespacing

\maketitle
\newpage
\tableofcontents
\newpage


\section{Uvod}
Algoritem sledenje žarku (\textit{ang. Ray Tracing}) je priljubljen algoritem, uporabljen v računalniški grafiki, ki simulira realistično osvetlitev prizorov. Temelji na ideji sledenja žarka svetlobe po prizoru, izračunavanju njegovega preseka z objekti in odbijanju v drugo smer. Običajno ta algoritem uporabljamo v običajnem evklidskem prostoru, saj predstavlja resnični svet. Vendar pa lahko algoritem implementiramo tudi v drugih (neevklidskih) prostorih, da dobimo zanimive vizualne rezultate. To bo glavna naloga tega projekta.

\section{Sledenje žarkom}
"Sceno" predstavimo z enačbami ravnin oblike:
\begin{equation}
f(x, y, z) = 0. bla bla
\end{equation}
\section {Implementacija osnovnega algoritma ray tracing}
Implementirali smo osnovni algoritem sledenja žarkov. Naš program je sposoben upodobiti preprost prizor z osnovnim senčenjem. 
A SMO TO KEJ DODAL AL NE ???Vse druge razširitve, kot so odsevi in mehko senčenje, so izbirne, saj niso v ospredju tega projekta. ?????

\subsection{Metoda Intersection}
Raylaga kode 


\section{Implementacija algoritma sledenje žarkom za tri dimenzionalni ploščati torus} 
T3
f .
\section{Implementacija algoritma sledenje žarkom za 2-sphere}
Suggestion: first just try to plot one geodesic
using the method described above and then move to ray tracing.
\section {Primerjava metode sledenje žarku v evklidskem in neevklidskem prostoru}
What are the characteristics of each space? If you find any
interesting behaviours report on them and try to explain them

\section{Razdelitev dela}
\begin{itemize}
  \item Blaž Bergant: DOPIŠI SAM
  \item Martin Jereb: ogrodje poročila, implementacija razreda Torus, priprava PowerPoint predstavitve
  \item Matjaž Pogačnik: DOPIŠI SAM 
\end{itemize}


\section{Viri}
\begin{itemize}
  \item Kovač, G. (2023). \textit{Sledenje žarku v neevklidskih
prostorih} (Diplomska naloga). Ljubljana: [G. Kovač].
  \item Kovač, G. (2024). \textit{Ray tracing in Non-euclidean spaces}. Ljubljana: [G. Kovač].
\item Zalar, A. \textit{Mathematical Modelling, Lecture Notes} (2024). Ljubljana: [A. Zalar].\textit{}
\end{itemize}

\section{Priloga}
TUKI PRIDE KODA AL PA SLIKE AL KARKOL K NE BO PASAL V POROČILO

\end{document}
